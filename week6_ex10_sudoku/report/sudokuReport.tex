% !TeX spellcheck = en_GB
\documentclass[a4paper]{article}
%\documentclass[a4paper]{scrartcl}
\usepackage[utf8]{inputenc}
\usepackage[british]{babel}
\usepackage{fullpage,amsmath,float,graphicx,parskip,fancyhdr,textcomp,csquotes,xcolor, geometry,gensymb}
\setlength{\headheight}{12pt}
\setlength{\headsep}{25pt}
\usepackage[margin=10pt, font=small, labelfont=bf, labelsep=endash]{caption} %mooiere captions
\pagestyle{fancy}
\fancyhf{}
\fancyhead[L]{\bfseries\leftmark} %places section number and name in top left
\fancyhead[R]{\bfseries\thepage}%places pagenumber in top right
\usepackage[hidelinks]{hyperref}
\hypersetup{
	colorlinks,
	linkcolor={red!20!black},
	citecolor={blue!50!black},
	urlcolor={blue!80!black}	}

\title{\textbf{Sudoku solver with simulated annealing} \\\large{Exercise 10, Modelling and Simulation, Utrecht University}}
\author{Jim Carstens (5558816)}
\date{\today}


\begin{document}
\maketitle
\hrulefill
\tableofcontents
\hrulefill
%%%%%%%%%%%%%%%%%%%%%%%%%%%%%%%%%%%%%%%%
\section*{Introduction}
In this report, 

%%%%%%%%%%%%%%%%%%%%%%%%%%%%%%%%%%%%%%%%
\section{Sudoku}  \label{sec:c}
The Sudoku

\section{Contradiction}
$\vec{R}$ can be time independent if $\frac{d\vec{R}}{dt}=0$. This is the case when $\vec{\Omega} \times \vec{R}=0$. This holds for $|\vec{\Omega}||\vec{R}|\sin(\theta)=0$. Now, either $\vec{R}=0$, which is trivial. Or, $\vec{\Omega}=0$, in this case there is no optical field. Or, $\sin(\theta)=0$, in which case $\vec{\Omega}$ and $\vec{R}$ are parallel or point in the same direction.
It follows that it is possible for this equation to also yield a time-independent solution, either in absence of an optical field, or when the two vectors are parallel, solving the contradiction.







%%%%%%%%%%%%%%%%%%%%%%%%%%%%%%%%%%%%%%%%
%\appendix{}
%\section{}\label{}


\end{document}